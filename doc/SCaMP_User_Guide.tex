\documentclass[a4paper,10pt]{article}
\usepackage{graphicx}
\usepackage{fullpage}
\usepackage{fancyhdr}
\usepackage{fourier}
\usepackage{framed}
\usepackage{booktabs}
\usepackage[hyphens]{url}
\usepackage{hyperref}
\usepackage{array}
\usepackage[justification=justified,singlelinecheck=false]{caption}
\usepackage{float}
\usepackage{todonotes}
\usepackage{subcaption}
\usepackage{natbib}

\setlength{\headsep}{20pt}
\setlength{\headheight}{12pt}
\pagestyle{fancy}
\setlength{\parskip}{10pt plus 1pt minus 1pt}
\sloppy

\newenvironment{tight_enumerate}{
\begin{enumerate}
  \setlength{\itemsep}{0pt}
  \setlength{\parskip}{0pt}
}{\end{enumerate}}

\newenvironment{tight_itemize}{
\begin{itemize}
  \setlength{\itemsep}{0pt}
  \setlength{\parskip}{0pt}
}{\end{itemize}}


\begin{document}
\begin{titlepage}
\begin{center}
  \bfseries
  \huge SCalable Metagenomics Pipeline (SCaMP) User Guide
  \vskip 0.1in
  \textsc{\normalsize Version 0.10 }
  \vskip 0.1in
  \textsc{\normalsize James Abbott (j.abbott@imperial.ac.uk)}
\end{center}

\tableofcontents
\renewcommand{\baselinestretch}{1.0}\normalsize
\end{titlepage}
\newpage
\graphicspath{ {images/} }

\section{Introduction}

The SCalable Metagenomics Pipeline (SCaMP) is a system for high-throughput
analysis of shotgun metagenome samples. It combines many tools (selected as
being the most effective in our evaluations) to determine community
composition, gene representation and abundance through metagenome assembly and
annotation and pathway representation. 

\section{Installation}

SCaMP requires a number of prerequisite packages to be installed, in addition                            
to the SCaMP software itself. This can either be achieved by installing all the
required packages and Perl modules manually, or through the use of conda, which
is the recommended approach since it greatly simplifies installation.  

\subsection{Quick installation with Conda (Recommended)}

You will need to first install both git and miniconda, and setup conda channels
as described on the bioconda installation page: \url{https://bioconda.github.io/#install-conda}.                                  

\subsubsection{SCaMP installation}

SCaMP can be then be downloaded from the git repository using the command:                                           

{\tt git clone https://github.com/jamesabbott/SCaMP.git}

Once the repository is cloned, the prerequisite packages can be installed from
conda using the {\tt setup.sh} script:           

{\tt SCaMP/bin/setup.sh}

The setup script will create a new conda environment name 'SCaMP' to avoid
conflicts with any existing conda installations, and use this to install
the prerequisite packages. Once completed, the script will give the option of
appending the SCaMP bin directory to your path. This will ensure the SCaMP
commands are readily available from the command line.

\subsubsection{Customising the SCaMP Environment}

If your system requires specific configuration such as manually setting a path,
or loading an environment module to access conda, then the {\tt bin/SCaMP}
script can be edited to include any necessary settings. See the comments at the
top of the script for examples of the kind of configuration which may be
configured. The {\tt setup.sh} script will have uncommented a line in this
section to activate the SCaMP conda environment each time the software is run.

\subsection{Manual Installation (The Slow Way...)}

% comment out conda lines in SCaMP
\subsubsection{Perl Modules}

A number of non-standard perl modules need to be installed prior to installing SCaMP: 

\begin{tight_enumerate}
\item BioPerl                                          
\item Bio::DB::EUtilities                              
\item DateTime                                         
\item File::Copy::Recursive                            
\item File::Find::Rule                                 
\item HTML::Entities                                   
\item LWP                                              
\item LWP::Protocol::https                             
\item Net::FTP::Recursive                              
\item Parallel::ForkManager                           
\item XML::Simple                                     
\item YAML::XS            
\end{tight_enumerate}

Most of these will be available through your systems package management system
(using {\tt yum} or {\tt apt}), however others may require installation from
CPAN.  

\subsubsection{Prerequisite packages}

A number of software packages also need to installed, and made available on the
system path prior to running SCaMP. These are listed in table
\ref{tab:reqpack}, along with tested version numbers.

\begin{table}[htb]
\begin{tabular}{ll}
\hline
\textbf {Package} & \textbf {Tested Version}  \\
\hline
trim-galore        & 0.45 \\
R                  & 3.3.2 \\
\hline
\end{tabular}
\captionof{table}{Required software packages}
\label{tab:reqpack}
\end{table}

\subsubsection{SCaMP installation}

SCaMP can be now be downloaded from the git repository using the command:                                           

{\tt git clone https://github.com/jamesabbott/SCaMP.git}

\subsubsection{Customising the SCaMP Environment}

If your system requires specific configuration such as manually setting a path,
or loading an environment module to access conda, then the {\tt bin/SCaMP}
script can be edited to include any necessary settings. See the comments at the
top of the script for examples of the kind of configuration which may be
configured.

\section{Running SCaMP}

\subsection{Software Layout}

The SCaMP software is layed out across a few directories, the contents of which are hopefully self-explanatory:

{\tt bin} - Executable scripts:
\begin{tight_itemize}
\item \textbf{build\_ref\_db}: Downloads and indexes reference databases.
\item \textbf{filter\_reads}: Filters sequence reads against reference database.
\item \textbf{qc\_trim}: Runs QC on sequence reads and applies quality and adapter trimming.
\item \textbf{SCaMP}: Main script for launching analysis runs.
\item \textbf{setup.sh}: Installs prerequisite packages and configures system.
\end{tight_itemize}

{\tt doc} - Documentation
\begin{tight_itemize}
\item \textbf{SCaMP\_User\_Guide.tex}: LaTeX source for user guide.
\item \textbf{SCaMP\_User\_Guide.pdf}: PDF format user guide.
\end{tight_itemize}

{\tt etc} - Configuration data
\begin{tight_itemize}
\item \textbf{barcodes.txt}: Example barcode sequences for adapter trimming.
\item \textbf{conda\_packages.txt}: List of conda packages to be installed by {\tt bin/setup.sh}.
\item \textbf{SCaMP.yaml}: YAML format configuration file.
\end{tight_itemize}

{\tt lib} - Common software components used by scripts in {\tt bin}
\begin{tight_itemize}
\item \textbf{SCaMP.pm}: Class of common methods used by scripts in {\tt bin}.
\end{tight_itemize}

\subsection{Preparing to Run SCaMP}

\subsubsection{Configuration File}

The {\tt etc/SCaMP.yaml} defines the SCaMP configuration. In shared
installations, each user can copy this file to a file named {\tt .SCaMP.yaml}
in their home directory. If this file is found, then the configuration will be
read from this file in preference to the {\tt etc/SCaMP.yaml} file.

The configuration file should be edited to define the following attributes:

\begin{tight_itemize}
\item \textbf{work\_dir}: Path to directory where SCaMP analysis will be carried out
\item \textbf{database\_dir}: Path to directory for storing reference databases
\end{tight_itemize}

A series of directories will be created in {\tt work\_dir} as the analysis
proceeds, storing various intermediate files and analysis results. 

In a shared software installation, a single set of reference databases can be
downloaded and stored in a centralised {\tt database\_dir} to reduce disk space
usage. 

\subsubsection{Sequence Reads}

A directory should be created in {\tt work\_dir} named {\tt reads}, and fastq
files (prefereably compressed using gzip) for the project should be copied into
this directory.

SCaMP is designed to be run using paired reads from an Illumina sequencer,
typically a MiSeq or HiSeq. In order for the reads to be identified correctly
they should be named in accordance with one of the following convention:

\textit{samplename(\_barcode)?(\_xxxxx)?\_R[12].f(ast)?q.gz}

The filename should start with the name of the sample, followed by an optional
barcode (typically six or eight bases). Additional details may optionally be
included in the next section of the filename, following which the filename must
include R1 or R2 (indicating read 1 or read 2 of the paired reads). The
filename should be suffixed with either .fastq.gz or fq.gz.

Examples of valid filenames include:
\begin{tight_itemize}
\item \textit{Sample-001\_R1.fastq.gz}
\item \textit{Sample-002\_R2.fq.gz}
\item \textit{Sample-003\_TAATCG\_L005.R1.fastq.gz}
\end{tight_itemize}

\subsubsection{Reference Databases}

\end{document}
